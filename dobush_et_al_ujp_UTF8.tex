\documentclass[fleqn,twoside,twocolumn,nofootinbib,showkeys]{revtex4} % Specifies the document class %,unsortedaddress
%\usepackage[sec]{ujp} % \usepackage[cyr]{ujp} for cyrillic
\usepackage[sec,doi]{ujp_UTF8} % \usepackage[cyr]{ujp_UTF8} for cyrillic

\begin{document}
	\title[Колонтитул: Назва роботи (3-5 слів)/Headline: paper title (3-5 words)]%колонтитул
	{НАЗВА РОБОТИ/PAPER TITLE}%
	\author{Прізвище автора 1/Name of Author 1}%1
	\affiliation{Bogolyubov Institute for Theoretical Physics, Nat. Acad. of Sci. of Ukraine}%інститут 1
	\address{14b, Metrolohichna Str., Kyiv 03143, Ukraine}%адреса 1
	\email{ujp@bitp.kiev.ua}%e-mail 1
	\author{Прізвище автора 2/Name of Author 2}%2
	\affiliation{Department of Mathematics and Computer science, Universiteit Antwerpen}%інститут 2
	\address{Middelheimlaan 1, Antwerpen 2020, Belgium}%адреса 2
	
	\udk{№ УДК/UDC} \razd{\secix}
	
	\autorcol{N.N.\hspace*{0.7mm}Author1, N.N.\hspace*{0.7mm}Author2, N.N.\hspace*{0.7mm}Author3 et al.}%
	
	\setcounter{page}{1}%
	
	\begin{abstract}
		Аннотація/Abstract
	\end{abstract}
	
	\keywords{Ключові слова/Keywords.}
	
	\maketitle
	
	\section{Вступ/Introduction}
	
	Текст роботи/Paper text...
	
	\section{Розділ/Section}
	
	Для всіх формул вирівнювання ліворуч./All formulas should be shifted to the left.
	
	Для формул з номером:/For numbered formulas:
	\begin{equation}
	\end{equation}
	Для формул без номерів:/For unnumbered formulas:
	\[
	\]
	
	Для рисунків:/For figures:%
	
	%Fig.~1
	\begin{figure}% figure* for wide figure, [h] [!] to change the placement
		\vskip1mm
		%\includegraphics[width=\column]{1_e}
		\vskip-3mm\caption{  }
	\end{figure}
	
	Для таблиць:/For tables:% Таблиця 5х4
	
	\begin{table}[b]
		\noindent\caption{Назва таблиці/Table title}\vskip3mm\tabcolsep4.5pt
		
		\noindent{\footnotesize
			\begin{tabular}{|c|c|c|c|c}
				\hline%
				\multicolumn{1}{|c}{\rule{0pt}{5mm}Назва 1-ї колонки/Title of 1st column}%
				& \multicolumn{1}{|c}{Назва 2-ї колонки/Title of 2nd column}
				& \multicolumn{1}{|c}{Назва 3-ї колонки/Title of 3d column}
				& \multicolumn{1}{|c}{Назва 4-ї колонки/Title of 4th column}
				& \multicolumn{1}{|c|}{Назва 5-ї колонки/Title of 5th column}\\[2mm]%
				\hline%
				\rule{0pt}{5mm}&&&&\\ % Приклад/Example: 1&2&3&4&5\\
				&&&&\\%
				&&&&\\[2mm]%
				\hline
			\end{tabular}
		}
	\end{table}
	
	\section{Висновки/Conclusions}
	
	\vskip3mm \textit{Подяка/Acknowledgement.}
	
	\begin{thebibliography}{99}
		
		\bibitem{RoSmo} C.~Rovelli, L.~Smolin. Spin networks and quantum
		gravity. \textit{Phys. Phys. D}  {\bf 52}, 5743 (1995). [DOI: https://doi.org].%1
		
		\bibitem{Sho90} B.W.~Shore. \textit{The Theory of Coherent Atomic Excitation}
		(Wiley,  1990) [ISBN: 978-0471524175].%2
		
		\bibitem{3}
		\bibitem{4}
		
		\begin{flushright}
			{\footnotesize Received 26.01.18}
		\end{flushright}
	\end{thebibliography}
	
	Якщо стаття англ. мовою/For papers in English:
	
	\vspace*{-5mm}\rezume{%Резюме укр. мовою
		А.А. Автор1, А.А. Автор2, А.А. Автор3 та ін.}{НАЗВА СТАТТІ УКР.
		МОВОЮ} {Текст резюме укр. мовою.}{\textit{К\,л\,ю\,ч\,о\,в\,і\,
			с\,л\,о\,в\,а:} ключові слова.}
	
	Якщо стаття укр. мовою/For papers in Ukrainian:
	
	\rezumang{%Резюме англ. мовою
		N.N. Author1, N.N. Author2, N.N. Author3 et al.}{PAPER TITLE IN
		ENGLISH}{Annotation text.}{\textit{Keywords:} keywords.}
	
\end{document}
